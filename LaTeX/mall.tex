\documentclass[11pt,a4paper, english, swedish]{article}
\pdfoutput=1

\usepackage{custom_as}

\graphicspath{ {figurer/} }

%%Drar in tabell och figurtexter
\usepackage[margin=10 pt]{caption}
%%För att lägga in 'att göra'-noteringar i texten
\usepackage{todonotes} %\todo{...}, \todolist

%%För att själv bestämma marginalerna. 
%\usepackage[
%            top    = 3cm,
%            bottom = 3cm,
%            left   = 3cm, right  = 3cm
%]{geometry}

%%För att ändra hur rubrikerna ska formateras
%\renewcommand{\thesection}{...}


\begin{document}
%\input{titlepages.tex}% <- extern titelsida

%%%%%%%%%%%%%%%%% vvv Inbyggd titelsida vvv %%%%%%%%%%%%%%%%%
\begin{titlepage}
\title{}
\author{}
\date{\today}

\maketitle

\pagenumbering{roman} %%Romersk sidnumrering
%\thispagestyle{empty} \pagestyle{empty} %%Ingen sidnumrering 

%%Ser till att vi får rätt rubrik på svenska
\iflanguage{swedish}{\renewcommand{\abstractname}{Sammandrag}}{}
\begin{abstract}


\end{abstract}
\newpage
\tableofcontents
\end{titlepage}

\pagenumbering{arabic}
\setcounter{page}{1}
%%%%%%%%%%%%%%%%% ^^^ Inbyggd titelsida ^^^ %%%%%%%%%%%%%%%%%


\section{}


\subsection{}


\newpage
%Ser till att det blir rätt namn i rubriken
\iflanguage{swedish}{
\renewcommand{\refname}{Källförteckning}%för documentstyle: article
\renewcommand{\bibname}{Källförteckning}%för documentstyle: report
}{}
\bibliographystyle{ieeetr}
\bibliography{referenser}%kräver en fil som heter 'referenser.bib'          

\clearpage
\appendix%Resets the section counter and changes it to Alph                 
\setcounter{page}{1} %Resets the pgenumbering                               
\renewcommand*{\thepage}{A\arabic{page}}%Changes the pagenubering to 'A...'
%Ser till at det blir rätt namn
\iflanguage{swedish}{\renewcommand{\appendixname}{Bilagor}}{}
\phantomsection{}%This one is needed to make 'Appendix' show up in the TOC
\addcontentsline{toc}{part}{\appendixname} %Makes 'Appendix' show up in the TOC
%Byter tillbaks till det gamla namnet
\iflanguage{swedish}{\renewcommand{\appendixname}{Bilaga}}{}




\end{document}





%% På svenska ska citattecknet vara samma i både början och slut.
%% Använd två apostrofer (två enkelfjongar): ''.


%% Inkludera PDF-dokument
\includepdf[pages={1-}]{filnamn.pdf} %Filnamnet får INTE innehålla 'mellanslag'!

%% Figurer inkluderade som pdf-filer
\begin{figure}\centering
\centerline{ %centrerar även större bilder
\includegraphics[width=1\textwidth]{filnamn.pdf}
}
\caption{}
\label{fig:}
\end{figure}

%% Figurer inkluderade med xfigs "Combined PDF/LaTeX"
\begin{figure}\centering
\resizebox{.8\textwidth}{!}{\input{filnamn.pdf_t}}
\caption{}
\label{fig:}
\end{figure}

%% Figurer roterade 90 grader
\begin{sidewaysfigure}\centering
\centerline{ %centrerar även större bilder
\includegraphics[width=1\textwidth]{filnamn.pdf}
}
\caption{}
\label{fig:}
\end{sidewaysfigure}


%%Om man vill lägga till något i TOC
\stepcounter{section} %Till exempel en 'section'
\addcontentsline{toc}{section}{\Alph{section}\hspace{8 pt}Labblogg} 

